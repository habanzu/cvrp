\documentclass{article}
\usepackage[utf8]{inputenc}
\usepackage{ngerman}
\title{Vortragsnotizen}
\author{Georg Meinhardt}
\date{19. Juli 2022}

\begin{document}
\maketitle
\section*{Allgemein}
\begin{itemize}
    \item Etwas zu trinken bereitstellen, ich habe Husten!
    \item BA heute morgen abgebeben!
    \item Neue Leute, Spagat zwischen Einfürhung und Auswertung
\end{itemize}

\section*{Branch-and-Price}
\begin{itemize}
    \item Master Problem erklären
    \item Vor dem Pricing Problem das generelle Prinzip erklären
    \item reduzierte Kosten erwähnen
\end{itemize}

\subsection*{Pricing Problem}
\begin{itemize}
    \item q-Routes erwähnen
\end{itemize}

\subsection*{ng-Pfade}
\begin{itemize}
    \item Intuition: Aller Pfade in der Nachbarschaft
    \item $\Delta = 2$, 2-cycle ist verboten, 3-cycle nicht
    \begin{itemize}
        \item der 3 cycle verläuft in der Nachbarschaft von z
        \item v-w-v wäre erlaubt
    \end{itemize}
    \item $\Delta = 4$, beide cycles sind verboten
\end{itemize}
\subsection*{Untere Schhranken}
\begin{itemize}
        \item Paper von Fukasawa erwähnen
\end{itemize}

\section*{Ergebnisse}
\begin{itemize}
    \item Erzählen, dass diese ganzen Sachen implementiert wurden.
    \item C++
    \item Qualitätsausage über die finalen Bounds
    % \item Laufen lassen mit hohem time limit, weil die finalen Bounds wichtig.
    % \item 24h
    \item Idee dieser Präsentation: Kurze Vorstellung der Resultate, alles sauber belegt, neugierig machen auf die ganze Arbeit
    \item Nun eine Instanz anschauen
\end{itemize}
\subsection*{Generelle Ergebnisse}
\begin{itemize}
    \item Lagrangian Bound ist tatsächlich sinnvoll
    \item Das ziemlich schnell, wegen Heuristiken
    \item Bounds bilden eine Hierarchie (und erklären)
    \item Bounds sind nicht monoton (und erklären)
    \item Anzahl der Iterationen eingehen
    \item Nun zur Laufzeit (nächste Folie)
    \item Zuerst ng20 erklären und warum die Linien mittem im Plot starten
    \item Dann explizit auf ng8 zu cyc2 und SPPRC eingehen
    \item Übergang: Erzählen Auswertung ESPPRC und Terminierung von ng20
\end{itemize}
\subsection*{Vergleich mit relativen Schwellenwerten}
\begin{itemize}
    \item Hierarchie erklären und sagen, wann sie gilt
    \item Annahme der Hierarchie der Laufzeit
    \item Parameter $It$ erklären (eventuell auch am Plot)
    \item Plots zeigen
    \begin{itemize}
        \item Gruppieren und Runden erklären
        \item Spearman's rank corellation coefficent erklären (auf ganzen Daten)
    \end{itemize}
    \item ng20 nicht vergessen!
\end{itemize}

\subsection*{Farley}
\begin{itemize}
    \item Nur auf dem E Datensatz
    \item Plot zeigen für SPPRC Farley
    \item Peformance: 4 von 11 über $10^4$section, 9 von 11 über 1000 s
    \item DAnn allgemeine Performance Evaluation nennen.
    \begin{itemize}
        \item 38 nicht finish Pricing Calls ingesamt
        \item 21 dnf
        \item 13 Farley besser
        \item 4 war Lagrangian besser (sehr eng)
    \end{itemize}
\end{itemize}

\end{document}
